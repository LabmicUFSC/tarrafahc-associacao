\documentclass[a4paper]{report}
\usepackage[brazil]{babel}
\usepackage[utf8]{inputenc}
\usepackage[T1]{fontenc}
\usepackage[overload]{textcase}
\usepackage{titlesec}
\usepackage{enumitem}
\setlist{noitemsep}
\setlist{nolistsep}
\setlist[enumerate]{label=\Roman* -}
\usepackage{indentfirst}
\usepackage{hyphenat}

% remover \clearpage entre capítulos
\makeatletter
\renewcommand{\chapter}{\thispagestyle{plain}%
                        \global\@topnum\z@
                        \@afterindentfalse
                        \secdef\@chapter\@schapter}
\makeatother

% usar numerais romanos nos capítulos
\renewcommand{\thechapter}{\Roman{chapter}}
\titleformat{\chapter}[display]{}{\hfill \\ \bf \centering \MakeTextUppercase\chaptertitlename\ \thechapter\ }{0pt}{\bf \centering \MakeTextUppercase}
\titlespacing{\chapter}{0pt}{0pt}{0pt}
\addto\captionsbrazil{\renewcommand{\chaptername}{CAPÍTULO}}

\newcounter{artigo}
\newcommand{\artigo}{%
    \stepcounter{artigo}%
    \hfill \\
    \textbf{
    Art. \theartigo
    \ifnum \theartigo>9
        .
    \else
        º
    \fi
    }
}
\newcounter{paragrafo}[artigo]
\makeatletter
\newcommand{\paragrafo}{%
    \stepcounter{paragrafo}%
    \immediate\write\@auxout{%
        \noexpand\global%
        \noexpand\@namedef{paragrafos-\thechapter-\theartigo}{\theparagrafo}}%
    \textbf{
    \ifcsname paragrafos-\thechapter-\theartigo\endcsname
        \ifnum \csname paragrafos-\thechapter-\theartigo\endcsname=1
            Paragráfo Único -
        \else
            \S \theparagrafo º
        \fi
    \else
        \S \theparagrafo º
    \fi
    }
}
\makeatother

\begin{document}

\begin{center}
 \textbf{\MakeTextUppercase{Estatuto Social da Associação do Tarrafa Hacker Clube}}
\end{center}



\chapter{DA DENOMINAÇÃO, PRAZO DE DURAÇÃO, SEDE E OBJETO SOCIAL}


\artigo % DA DENOMINAÇÃO
A Associação do Tarrafa Hacker Clube, neste estatuto designada, simplesmente,
como Tarrafa,
é uma associação de direito privado,
sem fins econômicos ou lucrativos, de caráter social, cultural, educacional,
técnico\hyp científico, organizacional, filantrópico, assistencial, promocional,
recreativo e genial, sem cunho religioso ou partidário, com a finalidade
de atender a todos que a ela se dirigirem, independente de classe social,
nacionalidade, sexo, orientação sexual, raça, cor ou crença religiosa.

\paragrafo % DO PRAZO DE DURAÇÃO
O Tarrafa foi fundado no dia 25 de Maio de 2013
e tem prazo de duração indeterminado.

\paragrafo % DA SEDE
O Tarrafa tem sede e foro na
Rua João Pio Duarte Silva, 602, bloco C apartamento 301 – Córrego Grande – CEP 88037-000,
na cidade de
Florianópolis do Estado de Santa Catarina.


\artigo % DAS PRERROGATIVAS
No desenvolvimento de suas atividades, o Tarrafa observará os princípios
da legalidade, impessoalidade, moralidade, publicidade, transparência,
economicidade, da eficiência e da instrumentalidade das formas, com as
seguintes prerrogativas:

\begin{enumerate}
\item Promover a cultura hacker, da tecnologia livre, da defesa da livre
      circulação do conhecimento, e de formas de apropriação de tecnologia
      para o pleno exercício da cidadania;
\item Atuar em prol da ciência, tecnologia e inovação;
\item Planejar e realizar atividades educacionais que promovam os princípios
      da interdisciplinaridade e a educação para a cidadania, baseados na
      apropriação de tecnologia;
\item Promover debates, encontros e atividades culturais para disseminar os
      valores do compartilhamento de conhecimento, recursos e tecnologias
      livres;
\item Representar seus associados judicial ou extrajudicialmente, na defesa
      dos princípios, prerrogativas e finalidades do Tarrafa; e
\item Celebrar atos formais de parceria, convênios, contratos, termos de
      ajuste e outros instrumentos legais com organismos e entidades
      nacionais, estrangeiras e internacionais, públicas e privadas.
\end{enumerate}

\paragrafo
Para cumprir suas finalidades sociais, o Tarrafa se
regerá pelas disposições contidas neste estatuto e, ainda, por um Regimento
Interno aprovado pela Assembleia Geral.


\artigo % DOS COMPROMISSOS DO TARRAFA
O Tarrafa se dedicara às suas atividades através de seus administradores e
associados, e adotará práticas de gestão administrativa, suficientes a coibir
a obtenção, de forma individual ou coletiva, de benefícios ou vantagens,
lícitas ou ilícitas, de qualquer forma, em decorrência da participação nos
processos decisórios, e suas rendas serão integralmente aplicadas em
território nacional, na consecução e no desenvolvimento de seus objetivos
sociais.



\chapter{DOS ASSOCIADOS}


\artigo % DOS ASSOCIADOS
Poderão filiar-se pessoas com idade maior ou igual a 18 (dezoito) anos,
ou maior ou igual a 16 (dezesseis) e menores de 18 (dezoito) anos legalmente
autorizadas, independente de classe social, nacionalidade, sexo, raça, cor,
orientação sexual, ou crença religiosa e, para seu ingresso, o interessado
deverá participar da lista de e-mails do Tarrafa.


\artigo % DOS DEVERES DOS ASSOCIADOS
São deveres dos associados:

\begin{enumerate}
 \item Cumprir e fazer cumprir o presente estatuto;
 \item Respeitar e cumprir as decisões da Assembleia Geral;
 \item Zelar pelo bom nome do Tarrafa;
 \item Defender o patrimônio e os interesses do Tarrafa;
 \item Cumprir e fazer cumprir o regimento interno;
 \item Exercer com dedicação, probidade, transparência e responsabilidade as
       funções para as quais forem indicados ou estiverem investidos; e
 \item Honrar pontualmente com suas contribuições associativas.
\end{enumerate}


\artigo % DOS DIREITOS DOS ASSOCIADOS
São direitos dos associados quites com suas obrigações associativas:

\begin{enumerate}
 \item Votar e ser votado para qualquer cargo da Diretoria Executiva ou do
       Conselho Fiscal, na forma prevista neste estatuto;
 \item Recorrer à Assembleia Geral contra qualquer ato da Diretoria ou do
       Conselho Fiscal;
 \item Comparecer por ocasião das eleições; e
 \item Votar por ocasião das eleições.
\end{enumerate}


\artigo % DA RETIRADA DO ASSOCIADO
É direito do associado retirar-se do quadro associativo, quando julgar necessário,
manifestando seu pedido junto à lista de e-mails do Tarrafa ou em ocasião de
reunião administrativa.


\artigo % DA EXCLUSÃO DO ASSOCIADO
A perda da qualidade de associado será determinada pela Assembleia Geral,
sendo admissível somente havendo justa causa, assim reconhecida em
procedimento disciplinar, em que fique assegurado o direito da ampla defesa,
quando ficar comprovada a ocorrência de:

\begin{enumerate}
 \item Violação do estatuto social;
 \item Atividades contrárias às decisões das assembleias gerais; e
 \item Conduta duvidosa, mediante a prática de atos incompatíveis com os
       preceitos do Tarrafa.
\end{enumerate}

\paragrafo
Definida a justa causa, o associado será devidamente notificado dos fatos a
ele imputados, através de notificação por e-mail, para que apresente sua
defesa prévia no prazo de 20 (vinte) dias a contar do recebimento da
comunicação.

\paragrafo
Após o decurso do prazo descrito no parágrafo anterior, independentemente da
apresentação de defesa, a representação será decidida em reunião
extraordinária da Assembleia Geral, por maioria simples de votos dos
associados presentes.

\paragrafo
Aplicada a pena de exclusão, caberá recurso, por parte do associado excluído,
à Assembleia Geral, o qual deverá, no prazo de 30 (trinta) dias contados da
decisão de sua exclusão, através de e-mail na lista do Tarrafa,
manifestar a intenção de ver a decisão ser objeto de deliberação, em última
instância, por parte da Assembleia Geral.

\paragrafo
Uma vez excluído, qualquer que seja o motivo, não terá o associado o direito
de pleitear indenização ou compensação de qualquer natureza, seja a que
título for.


\artigo % DA APLICAÇÃO DAS PENAS
As penas serão aplicadas pela Diretoria Executiva e estarão determinadas pelo
Regimento Interno.



\chapter{DA ASSEMBLEIA GERAL}


\artigo % DOS ORGÃOS DA INSTITUIÇÃO
A Assembleia Geral Deliberativa é o órgão máximo, soberano e excelso do
Tarrafa, e será constituída pelos seus associados em pleno gozo de seus
direitos.

\paragrafo
São órgãos do Tarrafa, independentes e harmônicos entre si, a Diretoria
Executiva e o Conselho Fiscal, estando eles diretamente subordinados à
Assembleia Geral.


\artigo % DA ASSEMBLEIA GERAL
A Assembleia Geral reunir-se-á no ínicio do ano em data deliberada através
da lista de e-mails,
para tomar conhecimento das ações da Diretoria Executiva e,
extraordinariamente, quando devidamente convocada. Constituirá em primeira
convocação com a maioria absoluta dos associados e, em segunda convocação,
três horas após a primeira, com qualquer número, salvo nos casos previsto
neste estatuto, tendo as seguintes prerrogativas:

\begin{enumerate}
 \item Fiscalizar os membros do Tarrafa, na consecução de seus objetivos;
 \item Eleger e destituir os administradores;
 \item Deliberar sobre a previsão orçamentária e a prestação de contas;
 \item Deliberar quanto à compra e venda de bens do Tarrafa;
 \item Aprovar o regimento interno, que disciplinará os vários setores de
       atividades do Tarrafa;
 \item Alterar, no todo ou em parte, o presente estatuto social;
 \item Deliberar quanto à dissolução do Tarrafa; e
 \item Decidir, em ultima instância, sobre todo e qualquer assunto de
       interesse social, bem como sobre os casos omissos no presente estatuto.
\end{enumerate}

\paragrafo
As assembleias gerais poderão ser ordinárias ou extraordinárias, e serão
convocadas, por qualquer Associado, mediante mensagem publicada na lista de
e-mails do Tarrafa onde constará: local, dia, mês, ano, hora da primeira e
segunda chamada, ordem do dia, e o nome de quem a convocou.

\paragrafo
Considerar-se-á convocada a Assembleia Geral no momento em que a quinta parte
dos associados se manifestar, em resposta à mensagem proposta por quem a
convocou, ciente da convocação e anuir ao debate acerca da matéria proposta
para deliberação.

\paragrafo
Toda e qualquer deliberação do Tarrafa será tomada por maioria de votos
dos Associados presentes, exceto aqueles que digam respeito à destituição de
administradores, alteração deste Estatuto Social e dissolução da Associação,
para as quais o quórum de deliberação é de dois terços de votos dos Associados
presentes à reunião.

\paragrafo
Toda e qualquer deliberação do Tarrafa será tomada por escrutínio público,
sendo vedada qualquer forma de deliberação por voto secreto, sob pena de
infração ao princípio da transparência sobre o qual se funda esta organização.


\chapter{DA DIRETORIA EXECUTIVA}


\artigo % DA DIRETORIA EXECUTIVA
A Diretoria Executiva do Tarrafa será constituída por 05 (cinco) membros,
os quais ocuparão os cargos de: Presidente, Vice-Presidente, Secretário,
Tesoureiro e Coringa. A Diretoria reunir-se-á, ordinariamente, uma vez por mês
e, extraordinariamente, quando convocada pelo Presidente ou pela maioria de
seus membros.


\artigo % COMPETE À DIRETORIA EXECUTIVA
Compete à Diretoria Executiva:

\begin{enumerate}
 \item Dirigir o Tarrafa, de acordo com o presente estatuto, e administrar o
       patrimônio social;
 \item Cumprir e fazer cumprir o presente estatuto e as decisões da Assembleia
       Geral;
 \item Representar e defender os interesses de seus associados;
 \item Elaborar o orçamento anual;
 \item Apresentar à Assembleia Geral, na reunião anual, o relatório de sua
       gestão e prestar contas referentes ao exercício anterior;
 \item Admitir pedidos de inscrição de associados; e
 \item Acatar pedidos de retirada voluntária de associados.
\end{enumerate}

\paragrafo
As decisões da diretoria deverão ser tomadas por maioria de votos, devendo
estar presentes, na reunião, a maioria absoluta de seus membros.


\artigo % COMPETE AO PRESIDENTE
Compete ao Presidente:

\begin{enumerate}
 \item Representar o Tarrafa, ativa e passivamente, perante os órgãos públicos,
       judiciais e extrajudiciais, inclusive em juízo ou fora dele, podendo
       delegar poderes e constituir procuradores e advogados para o fim que
       julgar necessário;
 \item Convocar e presidir as reuniões da Diretoria Executiva;
 \item Presidir as Assembleias Ordinárias e Extraordinárias;
 \item Juntamente com o Tesoureiro, abrir e manter contas bancárias, assinar
       cheques e documentos bancários e contábeis;
 \item Organizar relatório contendo o balanço do exercício financeiro e os
       principais eventos do ano anterior, apresentando-o à Assembleia Geral
       Ordinária;
 \item Contratar funcionários ou auxiliares especializados, fixando seus
       vencimentos, podendo licenciá-los, suspendê-los ou demiti-los; e
 \item Criar departamentos patrimoniais, culturais, sociais e outros
       que julgar necessários ao cumprimento das finalidades sociais, nomeando
       e destituindo os respectivos responsáveis.
\end{enumerate}

\paragrafo
Compete ao Vice-Presidente substituir legalmente o Presidente, em suas
faltas e impedimentos, assumindo o cargo em caso de vacância.


\artigo % COMPETE AO SECRETÁRIO
Compete ao Secretário:

\begin{enumerate}
 \item Redigir e manter, em dia, transcrição das atas das Assembleias Gerais
       e das reuniões da Diretoria Executiva;
 \item Redigir a correspondência do Tarrafa;
 \item Manter e ter sob sua guarda o arquivo do Tarrafa; e
 \item Dirigir e supervisionar todo o trabalho da Secretaria.
\end{enumerate}

\paragrafo
Compete ao Coringa substituir o Secretário, em suas faltas e impedimentos,
assumindo o cargo em caso de vacância.


\artigo % COMPETE AO TESOUREIRO
Compete ao Tesoureiro:

\begin{enumerate}
 \item Manter, em estabelecimentos bancários, juntamente com o Presidente, os
       valores do Tarrafa, podendo aplicá-los, ouvida a Diretoria Executiva;
 \item Assinar, em conjunto com o Presidente, os cheques e demais documentos
       bancários e contábeis;
 \item Efetuar os pagamentos autorizados e recebimentos devidos ao Tarrafa;
 \item Supervisionar o trabalho da tesouraria e da contabilidade;
 \item Apresentar ao Conselho Fiscal, os balancetes semestrais e o balanço
       anual; e
 \item Elaborar, anualmente, a relação dos bens do Tarrafa, apresentando-a,
       quando solicitado, à Assembleia Geral.
\end{enumerate}

\paragrafo
Compete ao Coringa substituir o Tesoureiro, em suas faltas e impedimentos,
assumindo o cargo em caso de vacância.



\chapter{DO CONSELHO FISCAL}


\artigo % DO CONSELHO FISCAL
O Conselho Fiscal, que será composto por ao menos três membros, tem por
objetivo, indelegável, fiscalizar e dar parecer sobre todos os atos da
Diretoria Executiva do Tarrafa, com as seguintes atribuições:

\begin{enumerate}
 \item Examinar os livros de escrituração do Tarrafa;
 \item Opinar e dar pareceres sobre balanços e relatórios financeiro e
       contábil, submetendo-os a Assembleia Geral Ordinária ou Extraordinária;
 \item Requisitar ao Tesoureiro, a qualquer tempo, a documentação
       comprobatória das operações econômico-financeiras realizadas pelo
       Tarrafa;
 \item Acompanhar o trabalho de eventuais auditores externos independentes; e
 \item Convocar Extraordinariamente a Assembleia Geral.
\end{enumerate}

\paragrafo
O Conselho Fiscal reunir-se-á ordinariamente, uma vez por ano, em sua maioria
absoluta, e extraordinariamente, sempre que convocado pelo Presidente do
Tarrafa, ou pela maioria simples de seus membros.



\chapter{DAS ELEIÇÕES E MANDATO}


\artigo % DAS ELEIÇÕES E MANDATO
As eleições para a Diretoria Executiva e Conselho Fiscal realizar-se-ão,
conjuntamente e anualmente, podendo seus membros ser reeleitos.


\artigo % DA PERDA DO MANDATO
A perda da qualidade de membro da Diretoria Executiva ou do Conselho Fiscal
será determinada pela Assembleia Geral, sendo admissível somente havendo
justa causa, assim reconhecida em procedimento disciplinar, quando ficar
comprovado:

\begin{enumerate}
 \item Malversação ou dilapidação do patrimônio social;
 \item Violação deste estatuto;
 \item Abandono do cargo, assim considerada a ausência não justificada
       em 03 (três) reuniões ordinárias consecutivas, sem expressa comunicação
       dos motivos da ausência, à secretaria do Tarrafa;
 \item Aceitação de cargo ou função incompatível com o exercício do cargo que
       exerce no Tarrafa; e
 \item Conduta duvidosa, mediante a prática de atos incompatíveis com os
       preceitos do Tarrafa.
\end{enumerate}

\paragrafo
Definida a justa causa, o diretor ou conselheiro será comunicado, através de
notificação por escrito, dos fatos a ele imputados, para que apresente sua
defesa prévia à Diretoria Executiva, no prazo de 20 (vinte) dias, contados do
recebimento da comunicação.

\paragrafo
Após o decurso do prazo descrito no parágrafo anterior, independentemente da
apresentação de defesa, a representação será submetida à Assembleia Geral
Extraordinária, devidamente convocada para esse fim, composta de associados
contribuintes em dia com suas obrigações associativas, não podendo ela deliberar
sem voto concorde de dois terços dos presentes, sendo em primeira
chamada, com a maioria absoluta dos associados e em segunda chamada, uma hora
após a primeira, com qualquer número de associados, onde será garantido o
amplo direito de defesa.


\artigo % DA RENÚNCIA
Em caso da renúncia do Presidente, o cargo será preenchido pelo
Vice-Presidente.
Em caso de renúncia de qualquer outro membro da Diretoria Executiva,
o cargo será preenchido pelo Coringa.

\paragrafo
O pedido de renúncia se dará por escrito, devendo ser protocolado na
secretaria do Tarrafa, a qual, no prazo máximo de 60 (sessenta) dias,
contado da data do protocolo, o submeterá à deliberação da Assembleia Geral.

\paragrafo
Ocorrendo renúncia coletiva da Diretoria e Conselho Fiscal, o Presidente
renunciante, qualquer membro da Diretoria Executiva ou, em último caso,
qualquer dos associados, poderá convocar a Assembleia Geral Extraordinária,
que elegerá uma comissão provisória composta por 05 (cinco) membros, que
administrará a entidade e fará realizar novas eleições, no prazo máximo
de 60 (sessenta) dias, contados da data de realização da referida assembleia.
Os diretores e conselheiros eleitos, nestas condições, complementarão o
mandato dos renunciantes.

\paragrafo
Em caso de renúncia do Diretor Coringa, um novo Coringa será apontado em
reunião pela atual Diretoria Executiva.


\artigo % DA REMUNERAÇÃO
Os membros da Diretoria Executiva e do Conselho Fiscal não perceberão nenhum
tipo de remuneração, de qualquer espécie ou natureza, pelo exercício das
atividades atribuídas à diretoria executiva e ao conselho fiscal do Tarrafa.


\artigo % DA RESPONSABILIDADE DOS MEMBROS
Os associados, mesmo que investidos na condição de membros da diretoria
executiva e conselho fiscal, não respondem, nem mesmo subsidiariamente,
pelos encargos e obrigações sociais do Tarrafa.



\chapter{DO PATRIMÔNIO SOCIAL}


\artigo % DO PATRIMÔNIO SOCIAL
O patrimônio do Tarrafa será constituído e mantido por:

\begin{enumerate}
 \item Doações, legados, bens, direitos e valores adquiridos e suas possíveis
       rendas e, ainda, pela arrecadação dos valores obtidos através da
       realização de eventos;
 \item Aluguéis de bens, imóveis e juros de títulos ou depósitos; e
 \item Prestação de serviços dentro das prerrogativas sociais do Tarrafa.
\end{enumerate}


\artigo % DA VENDA
Os bens móveis e imóveis poderão ser alienados, mediante prévia autorização de
Assembleia Geral Extraordinária, especialmente convocada para este fim,
devendo o valor apurado ser integralmente aplicado no desenvolvimento das
atividades sociais ou no aumento do patrimônio social do Tarrafa.



\chapter{DAS DISPOSIÇÕES GERAIS}


\artigo % DO EXERCÍCIO SOCIAL
O exercício social terminará em 31 de dezembro de cada ano, quando serão
elaboradas as demonstrações financeiras da entidade, em conformidade com as
disposições legais.


\artigo % DA REFORMA ESTATUTÁRIA
O presente estatuto social poderá ser reformado no tocante à administração, no
todo ou em parte, a qualquer tempo, por deliberação da Assembleia Geral
Extraordinária, especialmente convocada para este fim, composta de associados
em dia com suas obrigações associativas, não podendo ela deliberar sem voto
concorde de dois terços dos presentes, sendo em primeira chamada,
com a maioria absoluta dos associados e em segunda chamada, três horas
após a primeira, com qualquer número de associados.


\artigo % DA DISSOLUÇÃO
O Tarrafa poderá ser dissolvido, a qualquer tempo, uma vez constatada a
impossibilidade de sua sobrevivência, face à impossibilidade da manutenção de
seus objetivos sociais, ou desvirtuamento de suas finalidades estatutárias ou,
ainda, por carência de recursos financeiros e humanos, mediante deliberação de
Assembleia Geral Extraordinária, especialmente convocada para este fim,
composta de associados em dia com suas obrigações associativas, não podendo ela
deliberar sem voto concorde de dois terços dos presentes, sendo em
primeira chamada, com a totalidade dos associados e em segunda chamada, três
horas após a primeira, com qualquer número de associados.

\paragrafo
Em caso de dissolução do Tarrafa, liquidado o passivo, os bens
remanescentes serão destinados para outra entidade assistencial congênere,
com personalidade jurídica comprovada, sede e atividade preponderante neste
país e devidamente registrada nos órgãos públicos competentes.


\artigo
O Tarrafa não distribui lucros, bonificações ou vantagens a qualquer
título, para dirigentes, associados ou mantenedores, sob nenhuma forma ou
pretexto, devendo suas rendas ser aplicadas, exclusivamente, no território
nacional.


\artigo % DO HACKER

O Tarrafa entende como hacker:
\begin{enumerate}
 \item A pessoa que tem gosto em ter um entendimento profundo do funcionamento
       interno de sistemas, computadores, circuitos eletrônicos e redes
       informáticas;
 \item Hobbistas interessados em computação pessoal, hardware e eletrônica,
       adeptos da prática de projeto e construção amadora;
 \item Alguém que aplica o seu engenho para conseguir um resultado
       inteligente, rápido e eficiente; e
 \item Alguém que goste do desafio intelectual de superar ou contornar
       limitações criativamente.
\end{enumerate}

\artigo % DAS OMISSÕES
Os casos omissos no presente Estatuto serão resolvidos pela Assembleia Geral.

\vspace{1cm}

\hfill {\raggedright Florianópolis, 25 de Maio de 2013}

\vspace{3cm}

\begin{center}
\makebox[7cm]{\hrulefill}

Presidente

\vspace{3cm}

\makebox[7cm]{\hrulefill}

Advogado

Nome: José Vitor Lopes e Silva

OAB: nº 23700/SC
\end{center}

\end{document}
